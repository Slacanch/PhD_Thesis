Transcription of DNA into RNA intermediates constitutes the first step in gene expression.
Even minute changes in transcription patterns can upset the balance of many essential cellular constituents, generating a cascade of responses with significant repercussions on every biological process.
Because of this massive potential, transcription is one of the most finely regulated events in the cell and according to the \gls{sgd} \cite{cherry:2012:saccharomyces} Gene Ontology annotation, 1231 out of 6691 genes in \cer{} (18\%) can influence or directly take part in the transcriptional process.

In eukaryotes, three distinct RNA polymerases exist. 
\gls{pol1}: responsible for the transcription of \gls{rrna}; \gls{pol2}: responsible for the transcription of both protein coding genes and many non-coding RNAs; and \gls{pol3}: responsible mainly for the transcription of tRNAs and some \gls{rrna}.
Although products of \gls{pol1} and \gls{pol3} are by far the most abundant in the cell, \gls{pol2} is tasked with the production of an extremely varied set of transcripts and it is estimated that 80\% of the genome is actively transcribed by it \cite{david:2006:highresolution}. 
Because of this pervasiveness, transcription by \gls{pol2} must be tightly regulated to ensure its products are viable, as well as to prevent interference with other processes.
In this dissertation i will focus on how transcrition by \gls{pol2} is controlled---especially through transcription termination---and what its effects are on other DNA-based biological processes.

The first three chapters of the introduction to this work will describe the transcriptional process along its three main steps: initiation, elongation and termination. I will highlight the main molecular determinants that give rise to each phase, as well as mechanistically characterize the process when appropriate. 
Because of the relevance for the results that will be presented, I have devoted particular attention to transcription termination and described it in detail.
In chapter 4, i will talk about the transcriptional landscape of \cer{}; a look into the world of pervasive transcription, along with the mechanisms that control it. I will highlight the different classes of non-coding RNAs transcribed by \gls{pol2} as well as the quality control pathways that ensure their degradation.
In connection with the results of this dissertation, chapter 5 will discuss a particular class of transcription factors known as General Regulatory Factors. I will describe these factors in the context of their multiple functions, focusing on their chromatin remodeling capabilities and their function at gene promoters.
Finally, in chapter 6 I will consider the process of DNA replication and its interaction with transcription.
I will first put replication in its appropriate context by describing the structure of replication origins and the mechanics of the process itself. I will then discuss the available literature in regard to the effect of transcription on replication initiation and origin specification.

In the results part, i will outline three different projects. The first consists of the characterization of road-block termination, a novel termination mechanism for \gls{pol2}. Second, i will explore the interaction between transcription and DNA replication, with particular attention to the effect of transcription on origin usage. Finally, the last chapter will focus on NNS termination and how the components of the NNS complex contact their cognate binding sites in different contexts.

The results presented here were obtained using \cer{} as a model organism.
Therefore, the ensemble of data cited in this manuscript refers to this organism unless otherwise stated.





%Irrespective of the type of RNA polymerase, transcription is divided into three fundamental stages: initiation involves the assembly of an RNA polymerase on a promoter sequence and its interaction with general transcription factors; elongation occurs when the polymerase escapes the promoter and is actively transcribing the DNA template; finally, termination determines the end of the process and results in the disassembly of the elongation complex and the release of the transcript into the nucleus. 
%A number of specific factors, post-translational modifications of the polymerase, and structural elements such as chromatin can regulate each of these stages.


%The following sections will explore and characterize the major actors in the transcriptional process of \gls{pol2} in yeast, separating between the three phases outlined above.
%I will describe the initiation phase by talking about how regions of transcription initiation are defined in the genome, what are the essential components required to initiate transcription and the mechanics of the transition from initiation to elongation.
%In the next phase, elongation, i will describe the mechanics of elongation, how the polymerase deals with obstacles, and how it can coordinate different maturation steps according to its position.
%Finally, because of its importance for the work described in this manuscript, the phase of transcription termination will be split so that each termination pathway can be described in some level of detail; with particular attention to the recent rise of non-canonical termination pathways as a fail-safe mechanism to maintain proper gene expression across the genome.  