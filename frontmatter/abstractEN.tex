Transcription of DNA into RNA intermediates constitutes the first step in gene expression. During the last decade, several studies showed that about 80-90\% of the genome is transcribed, and that transcription can initiate almost anywhere. This process—known as pervasive transcription—represents a serious threat to proper gene expression as it has the potential to interfere with not only other transcription events, but any DNA-based process. Selective transcription termination is therefore a mechanism of paramount importance for genome transcriptome stability and correct regulation of gene expression. Here we describe road-block termination, a novel termination mechanism for RNA polymerase II that functions to limit pervasive transcription and buffer the consequences of readthrough transcription at canonical terminators in S.cerevisiae. We show that several transcription factors can elicit this termination and that a number of unexpected genomic loci are associated with it. Additionally, we explore the possibility that road-block termination might contribute to specification of replication origins.