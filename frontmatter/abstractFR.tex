La transcription de l’ADN en ARN constitue la première étape de l’expression d’un gène. Durant les dix dernières années, plusieurs études ont montré qu'environ 80-90\% du génome est transcrit et que la transcription peut démarrer presque partout. Ce phénomène, connu sous le nom de transcription envahissante, représente une menace sérieuse contre l'expression correcte du génome car il peut interférer non seulement avec d’autres évènements de transcription mais également  avec n’importe quel procédé impliquant l’ADN. Une terminaison sélective est donc un mécanisme de la plus haute importance pour la stabilité du génome et la correcte régulation de l'expression des gènes. Ici nous décrivons la terminaison road-block, un nouveau mécanisme de la terminaison par l'ARN polymerase II, qui a pour fonction de limiter la transcription envahissante et de limiter les conséquences d’une translecture au niveau des sites de terminaison canoniques de S.cerevisiae. Nous démontrons également que plusieurs facteurs de transcription peuvent entrainer cette terminaison et que certains sites génomiques y sont associés. De plus, nous explorons également la possibilité que ces terminaisons road-block puissent contribuer à rendre spécifiques les origines de réplication.