\chapter*{\textbf{Methods}}



\section*{Cell growth and UV Cross-Linking}

For all datasets, 2L of yeast cells expressing Rpb1-HTP tag were grown at 30°C to OD600=0.6 in DO-Trp (drop out without tryptophan). For Anchor Away strains, rapamycin was added at OD600=0.4 for two hours to a final concentration of 1µg/ml. For rna15 experiment, cells were grown at 25°C to OD600=0.6. 1 volume of media preheated at 30 or 37°C was added and cultures were incubated at 30 or 37°C for 1 hour and half. Cells were submitted to UV crosslink in a UV lamp system (Megatron) for 50 seconds. Cells were harvested by centrifugation, washed in cold PBS and resuspended in 2.4 volume / g of cells of TN150 buffer (50 mM Tris pH 7.8, 150 mM NaCl, 0.1\% NP-40 and 5 mM beta mercaptoethanol) supplemented with protease inhibitors (cOmplete™, Mini, EDTA-free Protease Inhibitor Cocktail). This suspension was flash frozen in droplets and cells were mechanically broken using the Mixer Mill MM 400 by doing 5 cycles of 3 minutes at 20 Hrz. Resulting powder was kept at -80°C. 

\section*{RNA-Protein Complexes Purification}

Powders were defrozen and the resulting extracts were treated for one hour at 25°C with DNase I (165U/g of cells) to solubilize the chromatin and then clarified by centrifugating 20’ at 20000g at 4°C. 
IgG and nickel purification steps were performed as previously described in the original protocol \cite{granneman:2009:identification} with minor changes: i) IgG purification steps has been realized using M-280 tosylactivated dynabeads (15mg per samples) coupled with rabbit IgG. ii) To increase the stringency, washes following both IgG and nickel purification steps were done at high salt concentration (1 M NaCl). 
After over-night binding on nickel columns, sequencing adaptors were added on the RNA as described in the original procedure. Because we work on nascent transcripts, some changes were required to fit with our purpose: For instance, since nascent transcripts contain a 3’ hydroxyl group that is protected inside the polymerase, we decided not to do the dephosphorylation step usually required because of the RNAse treatement that leads to 3’ phosphate transcripts. Also, we modified both 5’ and 3’ adaptors  in order to sequence RNA molecules from the 3’ end so that we can determine the exact position of the polymerase. The 3’ ligation has been realized with T4 rnl 2 truncated K227Q enzyme (NEB) instead of classical T4 RNA ligase. 
RNA-protein complexes were eluted in 400 µl of elution buffer (50mM Tris pH 7.8, 50mM NaCl, 150mM Imidazole, 0.1\% NP-40, 5mM beta mercaptoethanol). Eluates were concentrated in Vivacon® ultrafiltration spin columns 30 KDa MWCO to a final volume of 120 µl. A protein fractionation step was performed using the Gel Elution Liquid Fraction Entrapment Electrophoresis (Gelfree) system from Expedeon. RNA pol II containing fractions were treated with 100 µg of proteinase K in a buffer containing 0.5\% SDS. RNA was purified by two phenol chloroform isoamyl and one chloroform steps and precipitated in ethanol. RNA were reverse transcribed using the Reverse transcriptase IV (Invitrogen)


\section*{Library Preparation}

Between half to ¾ of the cDNA solution was used to performed multiple PCR reactions in a final volume of 25 µl using the following conditions: 0.4 µM of ech primers 0.2 mM dNTP, 2.5 U LA Taq DNA polymerase from Takara, 1X LA PCR Buffer II and 2 µl of cDNA per reaction with the programme: 95°C 2’, 95°C 30’’, 58°C 45’’, 72°C 1’, 72°C 5’. 
PCR were pooled and treated with 200 U of Exonuclease I (NEB) per milliliter of PCR reaction for 1 hour at 37°C. After Exonuclease I inactivation for 20’ at 80°C, DNA was purified on PCR clean up columns (NucleoSpin® Gel and PCR Clean-up, Macherey-Nagel) and sent to sequence using Illumina technology (NextSeq sequencing). 

\singlespacing
\section*{Dataset Processing}
\subsection*{CRAC}
\doublespacing

Samples were demultiplexed using the pyBarcodeFilter script from the pyCRAC utility suite \cite{webb:2014:parclip}. 
Subsequently, the 3’ adaptor is clipped with Cutadapt \cite{martin:2011:cutadapt} (-a TGGAATTCTCGGGTGCCAAGGATCTCGTATGCCGTCTTCTGCTTG –m 10) and the resulting insert is quality trimmed from the 3’ end using Trimmomatic rolling mean clipping \cite{bolger:2014:trimmomatic} (window size = 5, minimum quality = 25). 
At this stage the pyCRAC script pyFastqDuplicateRemover is used to collapse PCR duplicates and ensure each insert is represented only once. 
Each unique insert in our library is associated with a 6  nucleotides random tag within the 5’ adaptor. 
During Demultiplexing, pyBarcodeFilter retains this information in the header of each sequence. 
This information is used at this stage to better discern between identical inserts and PCR duplicates of the same insert. 
The resulting sequences are reverse complemented with Fastx\_reverse\_complement (part of the fastx toolkit, \url{http://hannonlab.cshl.edu/fastx_toolkit/}) and mapped to the R64 genome \cite{cherry:2012:saccharomyces} with bowtie2 (-N 1 –f) \cite{langmead:2012:fast}.

\subsection*{RNA-seq}

Samples are quality trimmed with trimmomatic (see above) and mapped to the R64 genome with bowtie2 with default options.

\section*{Metagene Analyses}

For each site included in the analysis we extracted the polymerase occupancy values corresponding to a 5kb window centered on the site. Each position (and its associated RNAPII occupancy value) in this window is then referenced by its relative distance to the ACS, ranging from -2500 to 2500. I calculated the median over all the values associated with the same relative position, and this median represents the value for that position in the final aggregate plot.

In certain cases, the median could not be used because of issues of depth. In this case we used the mean to summarize the data. When using the mean we eliminated extreme outliers to limit noise. For each relative position, we excluded from the analysis every value that was above the mean + 5 standard deviations calculated over all the values at that position.

\singlespacing


\begin{sidewaystable}


\resizebox{\textwidth}{!}{
\begin{tabular}[hp!]{ccccp{12cm}}%  
\hline 
Figure & N sites & Sites Origin & Dataset Origin & Notes \\ 
\hline 
\ref{fig:four}A & 272 & \cite{kubik:2015:nucleosome} & This Study & Only sites with a gene within 300bp downstream of the site were used \\ 
\hline 
\ref{fig:four}B & 294 & \cite{zentner:2015:checseq} & This study & we selected the top 500 sites of the "fast" category and analyzed in both orientation. only sites that had an annotated transcript within 300 bp upstream of the site were used. \\ 
\hline 
\ref{fig:four}C & 117 & \cite{zentner:2015:checseq} & This study & Only sites with a gene within 300 bp upstream of the site were analyzed \\ 
\hline 
\ref{fig:four}D & 161 & \cite{kubik:2015:nucleosome} & This study & Only sites with a gene within 300 bp upstream of the site were analyzed \\ 
\hline 
\ref{fig:five} & 161 & \cite{pelechano:2013:extensive}  & This Study & I considered the P(A) site of a gene the position with the highest TIF-seq signal within the first 500 bp downstream of the stop codon of that gene. P(A) sites with a Rap1 site within 300 bp downstream (rap1 sites defined according to \cite{kubik:2015:nucleosome}) were considered for this analysis.	
 \\ 
\hline 
\ref{fig:nine} & variable & \cite{macisaac:2006:improved} & \cite{schaughency:2014:genomewide} & sites for multiple DNA binding factors were considered. Metagene analyses were performed (see above) on all factors using WT data. \\ 
\hline 
\ref{fig:ten} & 16 & centromere annotation & \cite{schaughency:2014:genomewide} & NA	 \\ 
\hline 
\ref{fig:eleven} & 259 & tRNA annotation & \cite{schaughency:2014:genomewide} & NA \\ 
\hline 
\ref{fig:S6} & 117 & \cite{zentner:2015:checseq} & \cite{schaughency:2014:genomewide} & NA \\ 
\hline 
\ref{fig:S7} & 30 & \cite{zentner:2015:checseq} & This Study & we selected the top 500 sites of the "fast" category and analyzed in both orientation. only sites that had an annotated transcript within 300 bp upstream of the site were used.  \\ 
\hline 
\ref{fig:S9}Top & 12 & \cite{kubik:2015:nucleosome} & \cite{schaughency:2014:genomewide} & We considered sites with CUTs within 300 nucleotodes upstream and with a minimum average coverage of 5 in a 100 bp window upstream of the site.  \\ 
\hline 
\ref{fig:S9}Bottom & 152 & \cite{zentner:2015:checseq} & \cite{schaughency:2014:genomewide} & We considered sites with NNS and antisense transcripts within 300bp upstream of the Reb1 site \\ 
\hline 

\end{tabular}}
  \caption{Table of metagene analyses. N sites represents the number of sites used in the analysis. "Sites Origin" lists the origin of the sites used for the analysis. Dataset Origin lists the origins of the datasets used, "This study" is used when the datasets were generated in the lab. "Notes" adds details on the selection of the sites or usage of the datasets when needed.}
  \label{tab:test}
\end{sidewaystable}
\doublespacing

\section*{Boxplots in fig \ref{fig:five}C}

In order to represent the loss of polymerases due to termination, I calculated the average polymerase occupancy 100 nucleotide across the poly(A) site of all genes upstream of rap1 sites. This value was then divided by the average polymerase occupancy signal across the whole body of the gene. The overall distribution of these values for several datasets was compared with boxplots ant t-tests.


\singlespacing
\section*{Metagene Analysis of RNAPII Occupancy Around Origins}
\doublespacing

For each origin included in the analysis I identify the beginning of the ACS as the anchor point (the fixed reference around which all origins are aligned). 
I then extract the polymerase occupancy values corresponding to a 5kb window centered on the ACS for each of the assayed origins. 
Each position (and its associated RNAPII occupancy value) in this window is then referenced by its relative distance to the ACS, ranging from -2500 to 2500. 
I calculated the median over all the values associated with the same relative position, and this median represents the value for that position in the final aggregate plot.

To produce this plot, I used a wild type RNAPII parclip dataset \cite{schaughency:2014:genomewide}. I used a subset of 135 origins \cite{nieduszynski:2006:genomewide} that were surrounded by either convergent or tandem features \cite{xu:2009:bidirectional}. the final result was smoothed using the supsmu R function \cite{rproject} with a bandwidth of 0.01.

\section*{Metagene Analysis of Termination Events}

This metagene analysis was carried out as above, with minor differences. Instead of considering the full value associated with every position, I only considered whether there was a value or not. Presence of a value was considered as a 1, while absence was considered as a 0. Additionally, instead of calculating the median over all values present at the same relative position, I summed them and then divided this number by the total number of origins considered.

For this graph I used a polyadenylated 3’ end dataset \cite{wilkening:2013:efficient} and used 227 origins \cite{nieduszynski:2006:genomewide}.

\section*{Per-Origin Estimate of Sense and Antisense Transcription}

In order to calculate the amount of average transcription incoming towards the ACS in both the T-rich strand and the A-rich strand, I considered each strand for every origin and calculated the average RNAPII occupancy signal in a 100 bp window upstream of the ACS.

I used a wild type RNAPII parclip dataset for this purpose \cite{schaughency:2014:genomewide}.

\section*{Statistical Analysis of the Effect of Transcription on Replication Efficiency}

I obtained published per-origin estimates of licensing efficiency, timing efficiency and timing of firing \cite{hawkins:2013:highresolution}. 
We considered only origins for which 1) ACS annotation was present, 2) estimates of replication efficiency were available, and 3)  transcription levels could be calculated. 
A total of 190 origins were used for these experiments.

I use t-tests to compare the distribution and calculate p-values of different populations (boxplots). Correlations between populations were calculated with Pearson’s product moment correlation coefficient. The respective p-values were calculated with the appropriate correlation test (\texttt{cor.test()} in R).

\subsection*{Licensing Efficiency}
I divided the total of 190 origins in two equally populated sets according to their sense and antisense transcription levels. I then compared the licensing efficiencies of these two populations.
I decided to redo the experiment using only poorly licensed origins. I therefore eliminated all origins with licensing efficiency above 0.6. this left 43 origins.

\subsection*{Firing Efficiency}

In this experiment I selected efficiently and inefficiently firing origins according to firing efficiency relative to licensing efficiency. We normalized every firing efficiency by its own licensing efficiency and defined as efficient those origins that had a resulting score higher than 0.66, inefficient if lower.

\subsection*{Timing of Firing}

For this experiment I split the total of 190 origins into two halves according to transcription and compared the two population according to the distribution of their timing efficiencies .

\section*{SELEX and Artificial CUT Selection}

\paragraph{SELEX}
The SELEX experiment was performed in the lab by Jean-Baptiste Briand.
The 2000 most represented unique sequences of the final selected pool were kept for further analysis.
200,000 sequences were kept from the na\"{i}ve pool in order to calculate background distributions.

\paragraph{Artificial CUT Selection}
Artificial CUT selection data was performed by Odil Porrua in the lab \cite{porrua:2012:in}.
a total of 1000 sequences from the sequencing of the final selected pool was kept for analysis.
200,000 sequences were kept from the na\"{i}ve pool in order to calculate background distributions.



\section*{Algorithm for Motif Analysis}

In order to calculate the enrichment of a specific motif in the selected pool relative to the background nucleotide distribution of the na\"{i}ve pool of sequences, we decided to employ an algorithm proposed by J. van Helden \cite{vanhelden:1998:extracting}. 
 
Let $M$ be an RNA motif of length $l$. The frequency of this motif in the na\"{i}ve pool then is:

\begin{equation} \label{feaf}
F_{naive}(M)  =\dfrac{occ(M)}{\sum_{i=1}^{S} L_i - l + 1} = \dfrac{occ(M)}{T}
\end{equation}

Where $F_{naive}(M)$ is the frequency of M in the na\"{i}ve pool, $occ_(M)$ are the occurrences of $M$ over all na\"{i}ve sequences, $S$ is the total number of  sequences, and $L_i$ represents the length of the $i$th sequence. $T$, therefore, represents the total number of possible positions that can accomodate motif $M$ across all sequences in the pool. We will use the frequency $F_{naive}(M)$ as the probability of observing $M$ in the selected pool under the assumption that no selection has taken place.


The probability of observing exactly $n$ occurrences of $M$ in the selected pool is estimated by the binomial formula:

\begin{equation}
P(occ(M)=n) = \dfrac{T!}{(T-n)! \times n!} \times (F_{naive}(M))^n \times (1-F_{naive}(M))^{(T-n)}
\end{equation}

Consequently, the probability to observe $n$ or more occurrences of motif $M$ within the selected pool is:

\begin{equation} \label{pval}
\begin{split}
P(occ(M) >= n) & = \sum_{j=n}^{T}P(occ(M) = j)\\
& = 1-\sum_{j=0}^{n-1}P(occ(M) = j)
\end{split}
\end{equation}

Substituting the number of detected occurrences of $M$ in the selected pool  within \ref{pval} results in the probability of that number of occurrences emerging by chance given the nucleotide bias of the naive pool. 



On this basis, we can define a significance coefficient:

\begin{equation}
Sig = -Log_{10}[P(occ(M) >= n)]
\end{equation}

This coefficient was used to assess the enrichment of Nrd1 and Nab3 sites with spacers of different length.

\singlespacing
\section*{Comparison of Motifs Between SELEX and Artificial CUT Selection}
\doublespacing

Both the SELEX experiment and the artificial CUT selection follow the same selection principle. A pool of random sequences is subjected to cycles of selection according to variable criteria. The final pool can then be compared to the starting pool of sequences in order to determine enrichment or depletion of specific motifs. 

To compare enrichment for all motifs in the two experiments, i analyzed their starting and the final pool with Rsat \cite{medinarivera:2015:rsat}. I then plotted the z-scores for each motif in figure \ref{fig:selexComp}.
In order to determine significantly enriched motifs, i calculated p-values based on z-scores using the R environment:

\begin{verbatim}
pvalue = 2 * pnorm(-abs(zscore))
\end{verbatim}

I then corrected the p-values for multiple hypothesis testing using the Benjamini-Hochberg correction \cite{hochberg:1990:more}. After the correction, only motifs with a p-value lower than 0.001 were considered enriched or depleted.

\section*{Analysis of Nucleotides Flanking GUAG and GUAA}

In this experiment i wanted to assess the over- or under-representation of specific nucleotides flanking GUAG or GUAA in two datasets: the final pool of the SELEX experiment, and a pool of CUT sequences extracted from the genome. 

In order to accomplish this, i compared the frequency of specific nucleotides surrounding GUAG and GUAA with the overall frequency of these nucleotides within the pool. 
The Log$_2$ of these ratios is represented in figure \ref{fig:flanking}. 
In order to calculate the statistical significance of the enrichment/depletion, i calculated p-values based on the binomial distribution (binomial test).

\subsection*{Correlating the Results Obtained in the Two Datasets}

Because of the environment, the amount of selective pressure, and the higher number of sequences, enrichment scores for Nrd1 and Nab3 binding motifs are much higher in the SELEX experiment than in the artificial CUT selection. This constitute a problem when trying to apply classical clustering approaches in order to determine the similarity of flanking nucleotide enrichment patterns. These techniques rely on eucledian distance between patterns to determine similarity, which is heavily biased by the substantial difference in enrichment values.

I wanted to show that despite overall differences in the magnitude of enrichment scores, the patterns of flanking nucleotides enrichment and depletion hold well even across different datasets. I therefore decided to use Pearson's correlation as a measure of similarity. This ensures that scale is a non-factor in the assessment of overall similarity between the patterns.