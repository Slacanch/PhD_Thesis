\section{Elongation} %do these last!
After escaping the \gls{pic}, \gls{pol2} enters the phase of productive elongation.
During this phase, the polymerase travels along DNA, catalysing the addition of nucleotides to the growing RNA molecule that is being synthesised.
The simple addition of nucleotides, however, is not enough to qualify a mature transcript.
Several essential processing steps take place during transcription elongation and contribute to the production of fully formed transcripts.
Among these, the addition of the 5' cap, splicing, and addition of a poly(A) tail all rely on the presence of \gls{pol2} and the TEC \todominor{gls TEC} in order to be carried out properly.
The precise composition of the TEC is poorly understood. 
However, as \gls{pol2} progresses through the transcription unit, several complexes and co-factors are known to dynamically associate with it in order to enact the various maturation steps.  
Transcription elongation is therefore a highly regulated activity that coordinates several different processes to produce mature transcripts.
This regulation is enacted by the cell through several distinct mechanisms, such as the phosphorylation of the \gls{ctd} and the modification of histones.
These very same regulation mechanisms---along with important regulatory sequences---will eventually mark the end of transcription elongation and the transition to transcription termination.

\subsection{}

\subsection{The CTD and its modifications}

\subsection{Chromatin and its impact}

\subsection{3' processing}


% for RSC complex => Carey et al., 2006; Mas et al., 2009