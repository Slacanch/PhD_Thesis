\section{Elongation} %do these last!
After escaping the \gls{pic}, \gls{pol2} enters the phase of productive elongation.
During this phase, the polymerase travels along DNA, catalysing the addition of nucleotides to the growing RNA molecule that is being synthesised.
The simple addition of nucleotides, however, is not enough to qualify a mature transcript.
Several essential processing steps take place during transcription elongation and contribute to the production of fully formed transcripts.
Among these, the addition of the 5' cap, splicing, and addition of a poly(A) tail all rely on the presence of \gls{pol2} and the TEC \todominor{gls TEC} in order to be carried out properly.
The precise composition of the TEC is poorly understood. 
However, as \gls{pol2} progresses through the transcription unit, several complexes and co-factors are known to dynamically associate with it in order to enact the various maturation steps.  
Transcription elongation is therefore a highly regulated activity that coordinates several different processes to produce mature transcripts.
This regulation is enacted by the cell through several distinct mechanisms, such as the phosphorylation of the \gls{ctd} and the modification of histones.
These very same regulation mechanisms---along with important regulatory sequences---will eventually mark the end of transcription elongation and the transition to transcription termination.

\subsection{Elongation through chromatin}
Chromatin represents an extremely repressive barrier to any kind of DNA based process.
As I briefly touched upon in previous sections, chromatin components---histones---need to be actively dislodged from promoter regions in order to allow the \acrlong{pic} to assemble.
Elongating \gls{pol2} faces very similar problems, as in order to synthesise the RNA, it has to move through an array of nucleosomes without losing contact with DNA.
Although \invitro evidence has shown that \gls{pol2} can effectively elongate through a single nucleosome \todominor{ref}, the elongation complex alone is not enough to mediate transcription through multiple nucleosomes.

The TEC can overcome this problem by enlisting the help of several histone chaperones and chromatin remodeling complexes, as well as by exploiting post translational modifications of histones. 
The current model for transcription through nucleosomes posits that histones are not completely removed from DNA, but are instead partially destabilized as to allow \gls{pol2} to more easily transcribe through them.
The most notable actors in this phase are \gls{hats} such as Gcn5 and the \gls{fact} (\glsdesc{fact}) complex. 
\gls{hats} are posited to travel with the polymerase, depositing an acetyl group on histone tails.
This has the consequence of destabilizing intra-nucleosome interactions, resulting in a more relaxed chromatin structure.
Once histones are acetylated, \gls{fact}---also travelling with the polymerase---destabilizes the H2A-H2B dimer \footnote{Two of the four core components of a histone. Histones are composed of two H2A-H2B dimers and one H3-H4 tetramer arranged in a symmetrical structure. } \todominor{make a figure and link it in the footnote}, removing it and facilitating transcription through the remaining incomplete nucleosome structure. 

Because of the importance of chromatin in preventing spurious initiation, the composition, modifications, and overall structure of nucleosomes must be reset after the passage of \gls{pol2}. 
Specific histone chaperones such as Spt6, together with methil-transferases and \gls{hdacs}, are involved in this process.
First, Spt6 and other histone chaperones reconstruct a complete histone in the wake of transcribing \gls{pol2}.
Subsequently, methil-transferases such as Set2 act by methilating lysine 36 on histone H3. 
Although this modification---unlike acetylation---has no structural consequences on the organization of nucleosomes, it can act as a platform for recruitment of \gls{hdacs}.
The RPD3 complex has high affinity for H3K36 methilation and is recruited immediately after the passage of \gls{pol2} in order to remove the acetyl groups from histones and thus reset the structure of chromatin.

\subsection{Transcriptional pausing}
Nucleosomes do not represent the only obstacle to productive elongation.
A number of occurrences can prevent \gls{pol2} from elongating forward, such as DNA damage, misincorporation of a nucleotide, or collision with another DNA-bound protein.
While the cell has evolved complex---and often slow---mechanisms to deal with the more serious instances \footnote{DNA damage, among others, results in ubiquitinylation and degradation of the largest subunit of \gls{pol2}}, transcriptional pausing represents the first and common consequence to all the above-cited events. 
Because reversible pause-inducing events are relatively common during transcription, the cell evolved an all-purpose mechanism to quickly try to resolve pausing before the slower, more complex systems are called into action. 




Studies in \coli have shown that RNA polymerase does not rely onn the typical ``power stroke" generated by ATP \todominor{add atp gls} hydrolisis in order to move forward. 
The polymerase uses a refined brownian ratchet mechanism involving the strength of the RNA-DNA hybrid to prevent backward movement \citep{barnahum:2005:ratchet}.


\subsection{The CTD and its modifications}
\gls{pol2} and the elongation complex are pivotal elements in coordinating many of the co-transcriptional processes that contribute to the maturation of the nascent RNA.
In order to dynamically recruit all the necessary factors and complexes when they are needed, the largest subunit of \gls{pol2} has evolved an unstructured C-terminal domain composed, in \cer, of 26 repeats of the heptapeptide \ctdshort \footnote{\ctdlong in expanded nomenclature}.
This cluster of repeats can be differentially phosphorylated in different phases of transcription elongation, acting as a dynamically changing interaction surface for different co-factors. 



% for RSC complex => Carey et al., 2006; Mas et al., 2009