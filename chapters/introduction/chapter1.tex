\begin{savequote}[70mm]
Hail to the CRAC king, baby!
\qauthor{Drice Challal}
\end{savequote}

\chapter{Transcription}
%general intro on transcription

Transcription of DNA into RNA intermediates constitutes the first step in gene expression.
Even minute changes in transcription patterns can upset the balance of proteins, ribozymes, tRNAs and functional non-coding RNAs, generating a cascade of cellular responses with  significant repercussions on almost every biological process.
Because of this massive potential, transcription is one of the most finely regulated events in the cell and according to the \gls{sgd} \cite{cherry:2012:saccharomyces} gene ontology annotation, 1231 out of 6691 genes in \cer (18\%) can influence or directly take part in the transcriptional process.

In eukaryotes, three distinct RNA polymerases exist. \gls{pol1}, responsible for the transcription of ribosomal RNA, \gls{pol2}, responsible for the transcription of both protein coding genes and many non-coding RNAs, and \gls{pol3}, responsible mainly for the transcription of tRNAs. 

Irrespective of the type of RNA polymerase, transcription is divided into three fundamental stages: initiation involves the assembly of an RNA polymerase on a promoter sequence and its interaction with general transcription factors; elongation happens after the polymerase escapes the promoter and is actively transcribing the DNA template; finally, termination determines the end of the process and results in the disassembly of the elongation complex and the release of the transcript into the nucleus. 
A number of specific factors, post-translational modifications of the polymerase, and structural elements such as chromatin can regulate each of these stages.

The following sections will explore and characterize the major actors in the transcriptional process of \gls{pol2} in yeast, separating between the three phases outlined above.
I will describe the initiation phase by talking about promoter regions and and \gls{nfr}, assembly of the \gls{pic}, general and gene-specific transcription factors, as well as the act of promoter clearance.
In the next phase, elongation, i will talk about the composition of the elongation complex, the impact of \todominor{ctd as acronym} CTD phosphorylation and chromatin states, and the effects of pausing and backtracking.
Finally, the phase of transcription termination will be split so that each termination pathway can be described in some level of detail; with particular attention to the recent rise of non-canonical termination pathways as a fail-safe mechanism to maintain proper gene expression across the genome.   

\section{Initiation}

%controlled on different levels, spatially -> chromatin, core promoter elements
% temporally -> gene specific transccription factor
%intensity -> core promoter elements, gene specific transcription factors

%chromatin -> NFR helps expose DNA (example "Regulated displacement of TBP from the PHO8 promoter in vivo requires Cbf1 and the Isw1 chromatin remodeling complex"),generated by chr remodelers like RSC (madhani 2009, create nfr from nothing.) and SWI1/SNF, H2A.Z is deposited to borders of NFR by SWR1 (same family as swi1/snf, "ATP-driven exchange of histone H2AZ variant catalyzed by SWR1 chromatin remodeling complex.", other madhani paper) .




Initiation is the first step in any transcription event. 
It therefore needs to be accurate in when and where it occurs. 
Transcription initiation fundamentally relies on the assembly of the \gls{pic} (a super-complex 1.5 megadaltons in size  containing \gls{pol2} \cite{fazal:2015:real-time}) on chromatin.
The assembly of such complex is spatially defined by two elements: chromatin structure and core promoter elements.
Both contribute to limit the amount of spurious transcription by ensuring robust assembly of the \gls{pic} only in promoter regions.

\subsection{Chromatin structure}

Chromatin is a higher order structure that forms when DNA wraps around histones, proteins that can efficiently arrange loose DNA into compact structures.
The simplest unit of chromatin consists of ~140 nucleotides of DNA tightly wrapped around a histone, forming a nucleosome.
The organization of the genome around units of nucleosomes has a moltitude of consequences, not least of which is to sterically prevent DNA binding proteins from accessing DNA. 
As transcription relies on assembly of \gls{pol2} and the \gls{pic} on DNA to complete its initial phase, nucleosomes pose a considerable barrier to efficient initiation.
The insulation of DNA by nucleosomes has been harnessed by the cell and made into a regulatory mechanism that can spatially regulate where transcription initiates. 
To allow transcription factors to associate with DNA in promoter regions, these are always associated with an \gls{nfr}, an area of the genome where nucleosomes are depleted, leaving naked DNA available for binding.

%rewrite from here
%These sequence elements contain binding sites for several transcription factors that will eventually assemble into the \gls{pic}, 
%Another spatial constraint on the location of transcription initiation is chromatin. 
%Nucleosomes can occlude core promoter sequences and prevent assembly of the \gls{pic}.
%Several mechanisms are employed by the cell to generate \gls{nfr} at promoter regions and therefore ensure that \gls{pic} assembly can proceed unimpeded \cite{field:2008:distinct; jiang:2009:compiled}.

Temporally, the amount and timing of initiation at any specific promoter can be finely tuned by the binding of gene-specific transcription factors, that can act as either enhancers or repressors; modulating initiation efficiency either constitutively or in response to environmental effects. 

After assembly of the \gls{pic}, the complex melts the promoter DNA and performs several rounds of abortive transcription, eventually escaping the promoter and entering the elongation phase.
All these regulatory mechanisms ensure that initiation remains confined to promoter regions, preventing transcription from interfering with other DNA-based biological processes.


The core promoter region can stretch for up to 200 \gls{bp} upstream of the transcription start site. 
It contains numerous sequence elements that are known to interact with general transcription factors and promote \gls{pic} assembly. 
In contrast to metazoans, \cer 

There appear to be no universal core promoter elements that are absolutely required for proper gene expression \cite{butler:2002:RNA}.
On the contrary, each sequence element is present in some, but not all promoters; contributing to combinatorial gene regulation \cite{smale:2001:core}.

\todowarning{figure \ref{coreprom} is wrong, DPE is only found in drosophila+ organisms}

\begin{figure}[!ht]
  
  %\centering
    \includegraphics[width=\textwidth,height=\textheight,keepaspectratio]{figures/introduction/core_promoter}
    \label{coreprom}
    \caption{The diagram represents some of the sequence elements important for the recruitment of basal transcriptional machinery at promoter regions. TATA box is present at highly regulated genes and can function independently of the other motifs. BRE is usually located upstream of the TATA box motif and acts as platform for the recruitment of TFIIB, one of the general transcription factors. Inr  }
\end{figure}

A representation of the most important core promoter elements is depicted in figure \ref{coreprom}. 
%\subsection{Gene specific transcription factors}
%\subsection{chromatin states}
%\subsection{promoter clearance}

%\section{Elongation}
%\subsection{the elongation complex?}
%\subsection{CTD phosphorylation}
%\subsection{TFIIS and backtracking}
%\subsection{chromatin states}

%\section{Termination}
%\subsection{CPF}
%\subsubsection{torpedo model}
%\subsubsection{allosteric model}
%\subsubsection{combined model?}
%\subsubsection{fate of the transcripts}
%\subsection{NNS}
%\subsubsection{mechanistic models}
%\subsubsection{DNA elements}
%\subsubsection{fate of the transcripts}
%\subsection{non-canonical termination}
%\subsubsection{spt whatever, look it up on the review}
%\subsubsection{Roadblock termination}
\clearpage