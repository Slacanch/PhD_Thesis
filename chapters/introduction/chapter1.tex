\begin{savequote}[70mm]
Hail to the CRAC king, baby!
\qauthor{Drice Challal}
\end{savequote}

\chapter{Transcription}
%general intro on transcription

Transcription of DNA into RNA intermediates constitutes the first step in gene expression.
Even minute changes in transcription patterns can upset the balance of proteins, ribozymes, tRNAs and functional non-coding RNAs, generating a cascade of cellular responses with  significant repercussions on almost every biological process.
Because of this massive potential, transcription is one of the most finely regulated events in the cell and according to the \gls{sgd} \cite{cherry:2012:saccharomyces} gene ontology annotation, 1231 out of 6691 genes in \cer (18\%) can influence or directly take part in the transcriptional process.


Transcription is divided into three fundamental stages: initiation involves the assembly of an RNA polymerase on a promoter sequence and its interaction with general transcription factors; elongation happens after the polymerase escapes the promoter and is actively transcribing the DNA template; finally, termination determines the end of the process and results in the disassembly of the elongation complex and the release of the transcript into the nucleus. 
Specific factors and post-translational modifications, as well as structural elements such as chromatin, regulate each of these stages.


%transcription is an important process
 %affects the state of the cell
 %even though layers exsist, they act on the basal levels set by transcription
%transcription is finely regulated
%a number of factors contribute to the regulation
%transcription is divided into three foundamental stages
%summarize the stages
%understanding of these three stages is fundamental (?)
%strong conclusion? 

\section{Initiation}

\subsection{General transcription factors}
\subsection{Gene specific transcription factors}
\subsection{chromatin states}
\subsection{promoter clearance}

\section{Elongation}
\subsection{the elongation complex?}
\subsection{CTD phosphorilation}
\subsection{TFIIS and backtracking}
\subsection{chromatin states}

\section{Termination}
\subsection{CPF}
\subsubsection{torpedo model}
\subsubsection{allosteric model}
\subsubsection{combined model?}
\subsubsection{fate of the transcripts}
\subsection{NNS}
\subsubsection{mechanistic models}
\subsubsection{DNA elements}
\subsubsection{fate of the transcripts}
\subsection{non-canonical termination}
\subsubsection{spt whatever, look it up on the review}
\subsubsection{Roadblock termination}