\section{Termination} %do writing new stuff in the morning, fixing in the afternoon.
%reset in case acronyms are cited previously, this is their main paragraph and the acronym needs to be in long form.
\glsreset{cpf}
\glsreset{nns}
After its synthesis and maturation are complete, the nascent RNA molecule must be released from the DNA template, and the elongation complex must be disassembled and its components recycled.
In \cer{}, transcription termination is enacted by several widely different mechanisms.
Two predominating pathways terminate the vast majority of transcripts generated by \acrlong{pol2}: the \gls{cpf} pathway and the \gls{nns}  pathway. 
Both these mechanisms rely on short sequences on the nascent RNA---coupled with specific modifications on the \gls{ctd} of \gls{pol2}---to recruit specific factors and enact the disassembly of the elongation complex and the release of the transcript in the nucleus.

Owing to the imperfection of biological systems, the cell has evolved several additional non-canonical systems to terminate transcription. Should \gls{cpf} or \gls{nns} fail in their tasks, at least one transcription factor is able to act as road-block for \gls{pol2} and bring about transcription termination. Additionally, the \gls{sns} processing factor Rnt1 is able to act as an emergency termination factor in certain circumstances. \TODO{check correctness and expand pls.}

Transcription termination is also strictly intertwined with some steps of 3' end processing and maturation. 
The \gls{cpf} complex couples termination with a polyadenylation step and export competence of the RNA was shown to require this termination mechanism. \TODO{check and expand, again} 




\subsection{The CPF-CF pathway}
The \gls{cpf} pathway was the first termination mechanism described in \cer{} because of its association with the termination of protein-coding genes\TODO{ref?}\TODO{put section label in footnote}\footnote{Its activity can extend to certain kinds of non-coding transcripts as well (see SECTIONLABEL for details)}. 
\gls{cpf} termination is unique as it results in cleavage of the nascent RNA before termination occurs.
The site of cleavage is specified through sequence elements present on the nascent RNA and plays an important role in kickstarting the termination reaction.

There exist some controversy about how \gls{cpf} termination mechanistically occurs.
The literature proposes for two models that explain termination through the \gls{cpf} pathway.
The allosteric model argues that transcription through the cleavage site leads to conformational changes in the elongation complex, leading to destabilization of the complex and eventually termination.
On the other hand, the torpedo model posits that after cleavage, the uncapped 5' end of the polymerase-associated transcript is attacked by exonuclease Rat1, leading to the dismantling of the complex through destabilization of the ternary complex once Rat1 catches up with the polymerase.

Regardless of the model, the main actor of this termination mechanism is the \gls{cpf} complex, a large assembly of modular sub-complexes that act in concert to execute all the required steps. 
This complexity makes \gls{cpf} the most reliable, efficient, and precise termination mechanism in \cer{}.

\subsubsection{Assembly of the complex}
\TODO{figure}
Recruitment and initial assembly of the \gls{cpf} complex onto the nascent RNA is promoted by two mechanisms: interaction with specific sequences elements, and interaction with the polymerase \gls{ctd}.

A key component of the \gls{cpf} complex, Pcf11, contains a peptide sequence able to recognize the \gls{ctd}. This \gls{cid} is able to specifically recognize the \sert{}-phosphorylated version of the heptapeptide.
Given the nature of this \gls{ctd} modification---which is confined to the later stages of transcription---density of the \gls{cpf} complex around the polymerase is selectively increased where the complex is more likely to be needed for termination (i.e. at the 3' end of transcription units), facilitating the eventual binding of \gls{cpf} to the sequence elements on the nascent RNA.

Unlike in human, where the cleavage site is defined by a single highly conserved hexanucleotide sequence on the nascent RNA, Yeast \gls{cpf} complex recognizes a number of degenerate short sequences.
Two sub-complexes of \gls{cpf}, \gls{cf1a} and \gls{cf1b}, are responsible for the recognition of these sequences.
In particular, Rna15 and Hrp1 (components of \gls{cf1a} and \gls{cf1b} respectively) directly bind the nascent RNA.
Associated factors Rna14 and Pcf11 contribute to the assembly of the whole complex by interacting with \gls{pol2} and forming a scaffold that serves to tether the catalytic portion of the \gls{cpf} complex to the cleavage site.

The bulk of the catalytic activity of the \gls{cpf} complex is contained in the \gls{cpfa} sub-complex.
\gls{cpfa} directly contacts the cleavage site with its Ysh1 subunit and is responsible of the cleavage of the nascent RNA, one of the events that is thought to kickstart the termination reaction.
\gls{cpfa} also coordinates the polyadenylation reaction through the subunits Yth1 and Fip1. 
These factors recruit and tether to the complex the poly(A) polymerase Pap1, which will begin catalyzing the addition of a poly(A) tail after the transcript has been cleaved.

\subsubsection{The torpedo model}








\subsubsection{The allosteric model}

\subsection{The NNS pathway}



\subsection{Non-canonical termination pathways}