\section{Termination} %do writing new stuff in the morning, fixing in the afternoon.
%reset in case acronyms are cited previously, this is their main paragraph and the acronym needs to be in long form.
\glsreset{cpf}
\glsreset{nns}
After its synthesis and maturation are complete, the nascent RNA molecule must be released from the DNA template, and the elongation complex must be disassembled and its components recycled.
In \cer{}, transcription termination is enacted by several widely different mechanisms.
Two predominating pathways terminate the vast majority of transcripts generated by \acrlong{pol2}: the \gls{cpf} pathway and the \gls{nns}  pathway. 
Both these mechanisms rely on short sequences on the nascent RNA---coupled with specific modifications on the \gls{ctd} of \gls{pol2}---to recruit specific factors and enact the disassembly of the elongation complex and the release of the transcript in the nucleus.

Owing to the imperfection of biological systems, however, the cell has evolved several additional non-canonical systems to terminate transcription. Should \gls{cpf} or \gls{nns} fail in their tasks, at least one transcription factor is able to act as road-block for \gls{pol2} and bring about transcription termination. Additionally, the \gls{sns} processing factor Rnt1 is able to act as an emergency termination factor in certain circumstances. \TODO{check correctness and expand pls.}

Transcription termination is also strictly intertwined with some steps of 3' end processing and maturation. 
The \gls{cpf} complex couples termination with a polyadenylation step and export competence of the RNA was shown to require this termination mechanism. \TODO{check and expand, again} 




\subsection{The CPF-CF pathway}
The \gls{cpf} pathway represents one of the two main termination mechanisms present in \cer{}.
It is mainly associated with the termination of protein-coding genes, but its activity can extend to certain kinds of non-coding transcripts as well (see next section for details). \TODO{put section label here}
\gls{cpf} 



\subsection{The NNS pathway}



\subsection{Non-canonical termination pathways}