\section{Termination} %do writing new stuff in the morning, fixing in the afternoon.
%reset in case acronyms are cited previously, this is their main paragraph and the acronym needs to be in long form.
\glsreset{cpf}
\glsreset{nns}
After its synthesis and maturation are complete, the nascent RNA molecule must be released from the DNA template, and the elongation complex must be disassembled and its components recycled.
In \cer{}, transcription termination is enacted by several widely different mechanisms.
Two predominating pathways terminate the vast majority of transcripts generated by \acrlong{pol2}: the \gls{cpf} pathway and the \gls{nns}  pathway. 
Both these mechanisms rely on short sequences on the nascent RNA---coupled with specific modifications on the \gls{ctd} of \gls{pol2}---to recruit specific factors and enact the disassembly of the elongation complex and the release of the transcript in the nucleus.

Owing to the imperfection of biological systems, the cell has evolved several additional non-canonical systems to terminate transcription. Should \gls{cpf} or \gls{nns} fail in their tasks, at least one transcription factor is able to act as road-block for \gls{pol2} and bring about transcription termination. Additionally, the \gls{sns} processing factor Rnt1 is able to act as an emergency termination factor in certain circumstances. \TODO{check correctness and expand pls.}

Transcription termination is also strictly intertwined with some steps of 3' end processing and maturation. 
The \gls{cpf} complex couples termination with a polyadenylation step and export competence of the RNA was shown to require this termination mechanism. \TODO{check and expand, again} 




\subsection{The CPF-CF pathway}
The \gls{cpf} pathway was the first termination mechanism described in \cer{} because of its association with the termination of protein-coding genes\TODO{ref?}\TODO{put section label in footnote}\footnote{Its activity can extend to certain kinds of non-coding transcripts as well (see SECTIONLABEL for details)}. 
\gls{cpf} termination is unique as it results in cleavage of the nascent RNA before termination occurs.
The site of cleavage is specified through sequence elements present on the nascent RNA and plays an important role in kickstarting the termination reaction.

There exist some controversy about how \gls{cpf} termination mechanistically occurs.
The literature proposes for two models that explain termination through the \gls{cpf} pathway.
The allosteric model argues that transcription through the cleavage site leads to conformational changes in the elongation complex, leading to destabilization of the complex and eventually termination.
On the other hand, the torpedo model posits that after cleavage, the uncapped 5' end of the polymerase-associated transcript is attacked by exonuclease Rat1, leading to the dismantling of the complex through destabilization of the ternary complex once Rat1 catches up with the polymerase.

Regardless of the model, the main actor of this termination mechanism is the \gls{cpf} complex, a large assembly of modular sub-complexes that act in concert to execute all the required steps. 
This complexity makes \gls{cpf} the most reliable, efficient, and precise termination mechanism in \cer{}.

\subsubsection{Assembly of the complex}
\TODO{figure}
Recruitment and initial assembly of the \gls{cpf} complex onto the nascent RNA is promoted by two mechanisms: interaction with specific sequences elements, and interaction with the polymerase \gls{ctd}.

A key component of the \gls{cpf} complex, Pcf11, contains a peptide sequence able to recognize the \gls{ctd}. This \gls{cid} is able to specifically recognize the \sert{}-phosphorylated version of the heptapeptide.
Given the nature of this \gls{ctd} modification---which is confined to the later stages of transcription---density of the \gls{cpf} complex around the polymerase is selectively increased where the complex is more likely to be needed for termination (i.e. at the 3' end of transcription units), facilitating the eventual binding of \gls{cpf} to the sequence elements on the nascent RNA.

Unlike in human, where the cleavage site is defined by a single highly conserved hexanucleotide sequence on the nascent RNA, Yeast \gls{cpf} complex recognizes a number of degenerate short sequences.
Two sub-complexes of \gls{cpf}, \gls{cf1a} and \gls{cf1b}, are responsible for the recognition of these sequences.
In particular, Rna15 and Hrp1 (components of \gls{cf1a} and \gls{cf1b} respectively) directly bind the nascent RNA.
Associated factors Rna14 and Pcf11 contribute to the assembly of the whole complex by interacting with \gls{pol2} and forming a scaffold that serves to tether the catalytic portion of the \gls{cpf} complex to the cleavage site.

The bulk of the catalytic activity of the \gls{cpf} complex is contained in the \gls{cpfa} sub-complex.
\gls{cpfa} directly contacts the cleavage site with its Ysh1 subunit and is responsible of the cleavage of the nascent RNA, one of the events that is thought to kickstart the termination reaction.
\gls{cpfa} also coordinates the polyadenylation reaction through the subunits Yth1 and Fip1. 
These factors recruit and tether the poly(A) polymerase Pap1 to the complex, which will begin catalyzing the addition of a poly(A) tail after the transcript has been cleaved.

\subsubsection{The torpedo model}

After cleavage and release of the nascent RNA, the elongation complex has successfully accomplished its job in the transcriptional process and is ready to be disassembled.
The ``torpedo" model is one of the two main mechanistic models that describes the process by which the \gls{tec} is removed from the DNA template.
According to this model, cleavage represents the main termination signal for the \gls{cpf} complex, as it leaves an uncapped 5'-P on the transcript associated with the still transcribing elongation complex.
These unprotected 5' is the substrate of \FtoT{} exonucleases, a class of enzymes that are known to progressively degrade RNA polypeptides.
The \FtoT{} exonuclease Rat1 was discovered to be associated with the \gls{cpf} complex and is thought to attack the 5' moiety of the \gls{pol2}-associated transcript, starting a processivity race with \gls{pol2}.
Upon winning the race, Rat1 would destabilize the structure of the ternary complex within the polymerase, causing it to break apart and detach from the DNA template.

There are several lines of evidence that support this model for \gls{cpf} transcription termination.
Both Rat1 and its human homolog Xrn2 exhibit termination defects in model cases when mutated \citep{kim:2004:yeast, west:2004:human}.
Furthermore, Rat1 and its co-factor Rtt103 were found to be strongly associated with the 3' end of genes and in physical association with the \gls{cpf} complex \citep{kim:2004:yeast,luo:2006:role}, supporting the idea of a functional recruitment to zones of active transcription termination.
Homology studies found that homologs of Rtt103 in both humans and \cele{} have roles in transcription termination\citep{morales:2014:kub5hera, cui:2008:genes}.
Finally, recent mechanistic studies \invivo{} have demonstrated the kinetic competition between Rat1 and the elongation complex. By employing mutant polymerases that elongate faster or slower than the wild type version, the authors were able to show that slower polymerases result in earlier termination, consistent with the notion that Rat1 needs to physically catch up with the polymerase in order to elicit termination\citep{fong:2015:effects}.

At the same time, several reports argue against the torpedo model as sole effector of transcription termination.
\emph{In vitro} studies were unable to reproduce the termination effect observed \invivo{} using only Rat1 \citep{dengl:2009:torpedo}. More recent ventures re-attempted the \invitro{} approach with limited success \citep{park:2015:unraveling}, but managed to demostrate that Rat1 is able to terminate polymerases that are destabilized by nucleotide misincorporation.
Several additional mechanistic studies showed that the exonucleolytic activity of Rat1 is unable to mediate the release of the polymerase from the template \citep{luo:2006:role, pearson:2013:dismantling}.
Moreover, termination defects caused by Rat1 mutants were not associated with stabilization of the \gls{pol2}-associated transcript, arguing against the model.
Finally, recent genome-wide studies were unsuccessful in detecting a widespread effect of Rat1 human homolog (Xrn2) in transcription termination\citep{nojima:2015:mammalian}.


\subsubsection{The allosteric model}

The torpedo model relies exclusively on cleavage of the RNA as a trigger for transcription termination.
An alternative model argues that cleavage is a dispensable signal, and that termination can happen independently of this step.
This ``allosteric" model posits that after transcription of the cleavage site, \gls{pol2} loses a lot of factors that qualify the elongation complex as such.
The loss of these ``anti-terminator" factors---components of the elongation complex that would prevent termination from happening---would trigger conformational changes, destabilize the polymerase, and allow components of the \gls{cpf} complex itself to elicit the disassembly of \gls{pol2} from the template.

Several studies support this model. 
\gls{pol2} was shown to lose a number of associated elongation factors after reaching the 3' end \citep{kim:2004:transitions}.
In addition, the component of the \gls{cpf} complex Pcf11 was shown to be able to terminate the polymerase \invitro{} by binding the nascent RNA and the \sert{}-phosphorylated moiety of \gls{pol2} \citep{zhang:2005:ctddependent}.
Ulterior support to this last study was provided by the same authors two years later, when they discovered that Pcf11 is able to perform the same feat in drosophila\citep{zhang:2006:pcf11}.
Finally, a very recent study published on Molecular Cell was able to reconstitute transcription termination in an \invitro{} system in the absence of cleavage\citep{zhang:2015:polya}.

\subsubsection{A unified view of CPF-CF transcription termination}

As evidence for and against the two models piles up, a unified view that combines elements of both torpedo and allosteric model is taking shape.
While the effect of Rat1 on transcription termination (of at least some transcripts) is established, its role as main effector of \gls{cpf} termination has been repeatedly called into question.
Several studies have now described interdependencies between Rat1 and other subunits of the \gls{cpf} complex---notably Pcf11---and the perceived nature of Rat1 is shifting towards that of a molecular effector  that is integrated into a larger system.
Moreover, proof of principle that termination is possible without cleavage has been recently provided---albeit \invitro{}---casting further doubt on the role of Rat1 and its exonucleolytic activity on transcription termination.

Despite recent advances and the rise of a unified model for transcription termination, mechanistic details on the termination reaction and what prompts it are still sorely lacking.



\subsection{The NNS pathway}

\gls{nns} dependent transcription termination is the second wide-spread termination mechanism in \cer{}.
It sets itself a part from \gls{cpf} termination in a number of ways.
First and foremost, it relies on a completely different---and much smaller---set of proteins: the two RNA binding proteins Nrd1 and Nab3, together with the RNA-DNA helicase Sen1.
Because of the different molecular effectors, the termination mechanism---although still not fully elucidated---is appreciably different.

It does not result in cleavage of the RNA transcript, but instead it is the disassembly of the elongation itself that mediates the release of the transcript. 
As a consequence, the 3' end of the terminated transcript coincide with the termination site on DNA, making \gls{nns} termination substantially less precise than \gls{cpf}.
The \gls{nns} complex also recruits a completely different set of 3' end processing and maturation factors called \gls{tramp} (\glsdesc{tramp}), which drives post-transcriptional polyadenylation and leads to trimming or complete degradation of the transcript. 

\gls{nns} termination operates mainly on non-coding RNAs and acts in the very early stages of transcription.
Despite not being directly involved in the termination of protein-coding genes, it can play a role in the regulation of gene expression by acting as an attenuator (i.e. terminating some transcription events very early in the transcription cycle, preventing them from coming to completion.) or otherwise modulating the transcription of nearby non-coding RNAs.


\subsubsection{The NNS complex structural overview of the proteins}

The main molecular effectors of the \gls{nns} complex are the three protein Nab3, Nrd1 and Sen1.
Despite their number, they present a remarkably complex network of interaction between each other, \gls{pol2}, and the nascent transcript.

\paragraph{Nab3}

This factor was originally identified as a polyadenylated RNA binding protein.
Its structure contains a conserved \gls{rrm} that can contact specific sequence elements on the nascent RNA, a \gls{nim} necessary for the interaction with Nrd1, and two low complexity domains: an essential Glutammine/Proline region at the C-terminus, and a dispensable Aspartic/Glutammic acid region at the N-terminus. 

Biochemical experiments have shown that Nab3 forms a stable heterodimer with Nrd1 through the \gls{nim} and contacts the RNA as such. 
In addition, the structure of the \gls{rrm} has been solved, revealing the structural basis for the preference of the sequence UCUUG.
Finally, its Glutammine/Proline region---despite being generally unstructured---can assemble into amyloid structures; a feature that might be important for termination.


\paragraph{Nrd1}

Identified as part of the ``nuclear pre-mRNA downregulation" family of proteins, Nrd1 is the most abundant of the three members of the complex.
Its main features consist of an \gls{rrm} structure that allows it to contact the nascent RNA, a \gls{cid} domain that interacts specifically with the \serf{}-phosphorylated version of the \gls{ctd}, and a Nab3 interaction motif that allows it to form a stable heterodimer.

Nrd1's \gls{rrm} was shown \invivo{} to contact the consensus sequence GTAA/G.
Recent \invitro{} studies, however, have called this notion into question by showing that several other G-rich and A-rich sequences could be bound equally well.
This led to the theory that presence of other members of the complex could enhance binding specificity.

In addition to the RNA, Nrd1 can contact \gls{pol2} through its \gls{cid}. 
The specificity for \serf{}-phosphorylated \gls{ctd} is one of the determinants for the early activity of the \gls{nns} termination pathway. 
Dispensable for cell viability, the interaction with \gls{pol2} is thought to precede RNA binding and promote termination by increasing the local density of Nrd1-Nab3 heterodimer in the early phases of transcription.
This mechanism is very similar to that employed by \gls{cpf} through its subunit Pcf11---containing a \gls{cid} with high affinity for \sert{}-P.
 
Curiously, Nrd1 also contains a Glutammine/Proline region at the C-terminus, similarly to Nab3. 
Deletion of this region shows no growth or termination defects, but is synthetic lethal if combined with other aphenotipic mutations on Nab3 (our unpublished data).
The functional implications of this genetic interactions are still unknown. 

\paragraph{Sen1}

This extremely large (253kDa) and very low abundance (125 molecules per cell) protein is the only member of the \gls{nns} complex to have enzymatic activity.
Sen1 was characterized as a helicase of the SFI superfamily and very closely related to Upf1, a member of the \gls{nmd} pathway in the cytoplasm.
Unlike its close relative, Sen1 posesses a nuclear localization signal and acts in concert with the Nrd1-Nab3 heterodimer in promoting termination of small non-coding transcripts.
Genome-wide studies,however, have revealed activity at some coding genes, leading to the speculation that Sen1 exerts a regulatory function on these transcripts. \TODO{verify this, ref 54 reines}
The mechanism of recruitment of Sen1 to the termination site is still not fully clear, but two-hybrid studies have shown that Sen1 can interact with the polymerase itself.

Interestingly, some mutations of the human homolog of Sen1 (Senataxin) cause develpment of neurodegenerative diseases. 
Introduction of those same mutations in yeast Sen1 lead to termination defects.

\subsection{Non-canonical termination pathways}