\section{The transcriptional landscape of \cer{}} %check again this stuff. i'm not convinced

The rise of microarrays and next generation sequencing techniques has made the genome-wide exploration of the transcriptome possible. 
Early application of tiling arrays to the transcriptome of \cer{} showed that the genome is pervasively transcribed and RNA molecules can arise from almost any place \cite{neil:2009:widespread,xu:2009:bidirectional}. 
There are multiple reasons for this phenomenon.
Chief among all is the fact that the genome provides a very low barrier to transcription initiation.
Removing or displacing nucleosomes is often enough to allow basal levels of transcription to initiate, even without any of the regulatory sequences usually found at promoters.
Secondarily, studies have shown that yeast promoters do not posses an inherent direction dictated by sequence elements, but are bidirectional \cite{neil:2009:widespread,xu:2009:bidirectional}.

Promoter bidirectionality, and the general propensity of transcription to initiate spuriously, contribute to the generation of large quantities of non-functional RNAs, usually referred to as pervasive transcripts.
In order to prevent these RNAs from interfering with the physiological activities, the cell evolved strict RNA  \gls{qcs}, active both in the nucleus and in the cytoplasm.
The vast majority of pervasive transcripts are subjected to quality control, and are degraded shortly after their production.

\subsection{RNA quality control systems}

RNA \gls{qcs} are multisubunit complexes that rely on exonucleases to mediate the degradation of specific RNAs.
Targeting of transcripts to quality control (i.e. marking for degradation) occurs in several ways: it can be directly connected to the termination mechanism used to release the transcript (as in the case of \gls{nns} termination), or it can depend on certain features of the RNA, such as lack of 5' cap or long \gls{utr} length.
Thanks to these properties, \gls{qcs} are able to identify not only pervasive transcripts, but also aberrant RNAs originating from protein-coding or other functional loci.
\gls{qcs} can act both in the nucleus and in the cytoplasm, and compartment-specific system exist to cater to the specific needs. 

The exosome 
The exosome's activity depends on the subunit Dis3\footnote{additional details on the structure of the exosome and \gls{tramp} are presented in section \ref{secTramp}}, that possesses \TtoF{} exonuclease and endonuclease activity.
In the nucleus, the exosome is associated with a second \TtoF{} exonuclease: Rrp6. This 



\subsection{Types of pervasive transcripts}

Because of the rapid turnover dictated by quality control systems, pervasive transcripts are extremely difficult to detect in wild type cells.
Several studies found that knocking-out certain elements of the quality control system would affect the stability of some RNAs, making them appear in transcriptome analyses.
Over time, it became obvious that several classes of pervasive transcripts existed, each responding differently to the deletion of different members of the quality control system.



subsets of pervasive transcripts were differentially affected by different elements of the quality control system.


Knocking-out elements of quality control that would increase their stability, making them detectable.



 that several classes of pervasive transcripts exist, with different sensitivities to different branches of the quality control system.
%Deleting exosome subunits (such as Rrp6 or Dis3), \gls{nmd} effectors (such as Xrn1), or even transcription termination factors (Nrd1) can contribute to the stabilization of subsets of previously undetectable non-coding RNAs.


\paragraph{CUTs}

The first class of pervasive transcripts to be described, \gls{cuts} emerge from the deletion of the nuclear exosome co-factor Rrp6 \cite{wyers:2005:cryptic}.
\gls{cuts} are short transcripts (600-800 bp) that represent the main targets of \gls{nns} termination.
Because of this, \gls{cuts} are handed over to \gls{tramp} and the nuclear exosome shortly after termination, resulting in their rapid degradation. \TODO{refs to termination of cuts by nns.}

\paragraph{SUTs}



\paragraph{XUTs}


\paragraph{RUTs}


\paragraph{RUTs}



